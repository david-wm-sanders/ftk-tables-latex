% !TEX TS-program = pdflatex
\documentclass[12pt]{article}

% Package the packages
\usepackage[T1]{fontenc}
\usepackage[utf8]{inputenc}
\usepackage{lmodern}
\usepackage[a4paper, margin=0.75in]{geometry}
\usepackage{calc}
\usepackage{tabularx, makecell}
% For nicer lists inside cells...
\usepackage{enumitem}
\setlist{nosep}
% -

\begin{document}
\section{Artefacts}
\subsection{1}
\begin{table}[h!]
\centering
% \caption{An informative caption}
% \label{A specific label}
\newcolumntype{k}{>{\raggedright\arraybackslash}p{2.75cm}}
\newcolumntype{v}{X}
\newcolumntype{t}{>{\raggedright\arraybackslash}p{\textwidth - 4\tabcolsep - 2.75cm - 3\arrayrulewidth}}
% Force cells to align content top-left
\renewcommand\cellalign{tl}
\begin{tabularx}{1\textwidth}{|k|v|k|v|}
\hline
FTK No.  & \texttt{[FTK Item No.]}  & Evidence No.  & \texttt{\textbf{DSxx}}  \tabularnewline \hline
Filename  & \multicolumn{3}{t|}{\texttt{[name\_of\_file.ext]}}  \tabularnewline \hline
Hash (MD5)  & \multicolumn{3}{t|}{\texttt{[MD5 hash of the file]}}  \tabularnewline \hline
Hash (SHA1)  & \multicolumn{3}{t|}{\texttt{[SHA1 hash of the file]}}  \tabularnewline \hline
Date Created  & \multicolumn{3}{t|}{\texttt{[The time at which the file was created]}}  \tabularnewline \hline
Date Modified  & \multicolumn{3}{t|}{\texttt{[The time at which the file was last modified]}}  \tabularnewline \hline
Date Accessed  & \multicolumn{3}{t|}{\texttt{[The time at which the file was last accessed]}}  \tabularnewline \hline
Is deleted?  & \multicolumn{3}{t|}{\texttt{[Was the file deleted? Yes/No]}}  \tabularnewline \hline
Size (L/P)  & \multicolumn{3}{t|}{\texttt{[The logical/physical size of the file]}}  \tabularnewline \hline
File path  & \multicolumn{3}{t|}{\texttt{[C:\textbackslash full\textbackslash path\textbackslash to\textbackslash file]}}  \tabularnewline \hline
% Evidence Description  & \multicolumn{3}{t|}{\textit{Description of the evidence.}}  \tabularnewline \hline
% If using makecell, we must makecell all cells in row to get correct rendering and content alignment
\makecell[{{p{2.75cm}}}]{\raggedright Evidence Description}  & \multicolumn{3}{t|}{
\makecell[{{p{\textwidth - 4\tabcolsep - 3cm - 3\arrayrulewidth}}}]{
\textit{[Description of the evidence.]}\\
\textit{Forced line breaks by using} \texttt{\textbackslash\textbackslash} \textit{are possible.}\\\\
\textit{By extension, new paragraphs can be achieved with} \texttt{\textbackslash\textbackslash\textbackslash\textbackslash}.
}}  \tabularnewline \hline
% Technique used  & \multicolumn{3}{t|}{\textit{Technique used to discover the evidence.}}  \tabularnewline \hline
% If using makecell, we must makecell all cells in row to get correct rendering and content alignment
\makecell[{{p{2.75cm}}}]{\raggedright Technique used}  & \multicolumn{3}{t|}{
\makecell[{{p{\textwidth - 4\tabcolsep - 3cm - 3\arrayrulewidth}}}]{
\textit{[Technique used to discover the evidence.]}\\
\textit{Itemized and enumerated lists are usable inside these cells. They can be:}
\begin{itemize}
  \item \textit{short - or longer if a complicated step of the process requires a more in depth explanation!}
  \item \textit{simple or nested (for breaking complicated steps into sub-steps):}
        \begin{enumerate}[label=\emph{\roman*.}]
          \item \textit{sub-step 1 of a complicated process...}
          \item \textit{sub-step 2 of a complicated process...}
        \end{enumerate}
  \item \textit{sweet}
  % Remove extraneous vertical space...
  \vspace{-\baselineskip}
\end{itemize}
}}  \tabularnewline \hline
% Reason for selection  & \multicolumn{3}{t|}{\textit{Why was this file was selected as evidence?}}  \tabularnewline \hline
% If using makecell, we must makecell all cells in row to get correct rendering and content alignment
\makecell[{{p{2.75cm}}}]{\raggedright Reason for selection}  & \multicolumn{3}{t|}{
\makecell[{{p{\textwidth - 4\tabcolsep - 3cm - 3\arrayrulewidth}}}]{
\textit{[Why was this file was selected as evidence?]}
}}  \tabularnewline \hline
\end{tabularx}
\end{table}
\end{document}
